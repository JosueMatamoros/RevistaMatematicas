\documentclass{book}

\usepackage[spanish]{babel}
\usepackage{graphics}
%\usepackage{t1enc}
%\usepackage{amsmath}
\newcommand{\R}{ \mbox{$I \hspace{-1.3mm} R$} }

\begin{document}

\selectlanguage{spanish}

\chapter{Soluci\'on mediante series de potencias}

\section{Introducci\'on}

Hasta ahora hemos resuelto, principalmente, ecuaciones diferenciales lineales de orden dos o superior, con 
coeficientes constantes, a excepci\'on de la ecuaci\'on de Cauchy-Euler. Las aplicaciones en 
las que aparecen ecuaciones diferenciales con coeficientes variables surgen con mucha frecuencia en 
diversas \'areas de la ciencia. 

Una ecuaci\'on diferencial lineal de segundo orden con coeficientes variables, tan sencilla como 

\[
y^{\prime \prime} + x y=0,
\] 

no tiene soluciones elementales; podemos encontrar dos soluciones linealmente independientes representadas 
por series infinitas.

Como hemos enfatizado pocas ecuaciones diferenciales tienen soluciones que pueden expresarse expl\'icita o 
impl\'icitamente en t\'erminos de funciones elementales. A\'un, cuando las soluciones no puedan expresarse 
de esta forma, el problema de hallarlas no es del todo desesperanzado, existen algunas otras 
alternativas como los m\'etodos: gr\'aficos, num\'ericos y de series de potencias del cual hablaremos en 
breve. 

Una soluci\'on impl\'icita expresada en t\'erminos de funciones elementales es de menos utilidad que una 
serie de potencias o una soluci\'on num\'erica, esto, por ser expresiones m\'as complicadas para las cuales 
resulta muy dif\'icil expresar una variable en t\'erminos de la otra.


\section{M\'etodo de los coeficientes indeterminados} \index{soluci\'on por series}

Vamos a limitar nuestro estudio al enunciado y manejo del m\'etodo, sin entrar en los detalles te\'oricos 
del mismo. Recuerde que una serie de potencias representa a una funci\'on $f$ en un intervalo de 
convergencia $I$ y que podemos derivarla sucesivamente, para obtener series para 
$f^{\prime}$, $f^{\prime \prime}$,$f^{\prime \prime \prime}$, etc. Por ejemplo,


\begin{eqnarray*}
f(x) & = & a_0 + a_1 x + a_2 x^2 + a_3 x^3 + \ldots = \sum_{i=0}^{\infty} a_n x^n\\
f^{\prime}(x) & = & a_1 + 2 a_2 x + 3 a_3 x^2 + 4 a_4 x^3 + \ldots = \sum_{n=0}^{\infty} n a_n x^{n-1}  \\
f^{\prime \prime}(x) & = & 2 a_2 + 6 a_3 x + 12 a_4 x^2 + 20 a_5 x^3 + \ldots = \sum_{n=0}^{\infty} n \left( n-1 \right) x^{n-2}
\end{eqnarray*}

Los siguientes ejemplos muestran como aplicar el m\'etodo de las series de potencias a la soluci\'on de 
ecuaciones diferenciales. Iniciamos con un ejemplo muy simple, pero que nos har\'a entender la mec\'anica 
del m\'etodo.

\vspace{0.5 cm}

{\bf Ejemplo} \\
Usando series de potencias halle la soluci\'on de la ecuaci\'on $y^{\prime} - 2y = 0$.

\vspace{0.5 cm}

{\bf Soluci\'on} \\

\vspace{0.5 cm}

Supongamos que la soluci\'on se puede expresar como

\[
y = \sum_{n=0}^{\infty} a_n x^n
\]

Entonces, $y^{\prime}$ esta dada por

\[
y^{\prime} = \sum_{n=0}^{\infty} n a_n x^{n-1}
\]

Sustituyendo en la ecuaci\'on diferencial, obtenemos que

\begin{eqnarray*}
y^{\prime} - 2y & = & 0 \\
\sum_{n=0}^{\infty} n a_n x^{n-1} - 2 \sum_{n=0}^{\infty} a_n x^n & = & 0 \\
\sum_{n=0}^{\infty} n a_n x^{n-1} & = & 2 \sum_{n=0}^{\infty} a_n x^n \\
\end{eqnarray*}

Ahora debemos ajustar los \'indices de las sumas de forma que
aparezca $x^n$ en cada serie.

\[
\sum_{n=-1}^{\infty} \left(n + 1 \right) a_{n+1} x^n = 2 \sum_{n=0}^{\infty} a_n x^n
\]

Igualando los coeficientes correspondientes

\[
\left(n + 1 \right) a_{n+1} = 2 a_n \Rightarrow a_{n+1} = \frac{2a_n}{n+1}
\]

para $n \geq 0$.

Esta f\'ormula genera los siguientes coeficientes

\begin{eqnarray*}
a_1 & = & 2a_0 \\
a_2 & = & \frac{2a_1}{2}=\frac{2^2 a_0}{2} \\
a_3 & = & \frac{2a_2}{3}= \frac{2^3 a_0}{2 \cdot 3}= \frac{2^3 a_0}{3!} \\
a_4 & = & \frac{2a_3}{4} = \frac{2^4 a_0}{2 \cdot 3 \cdot 4} = \frac{2^4 a_0}{4!} \\
\vdots \\
a_n & = & \frac{2^n a_0}{n!} \\
\end{eqnarray*}

De donde obtenemos que la soluci\'on esta dada por 

\[
y = \sum_{n=0}^{\infty} \frac{2^n a_0}{n!} x^n = a_0 \sum_{n=0}^{\infty} \frac{\left(2 x \right)^n}{n!} = a_0 e^{2x}
\]


Aqu\'{\i} hemos usado la expansi\'on en series de potencias para la funci\'on exponencial 

\[
e^x = \sum_{n=0}^{\infty} \frac{x^n}{n!}
\]

{\bf Observaci\'on:} esta ecuaci\'on diferencial puede ser resuelta de manera m\'as simple por medio de 
separaci\'on de variables.

\[
y^{\prime} - 2 y = 0 \Rightarrow \frac{dy}{y} = 2 dx \Rightarrow Ln(y) = 2x + c  \Rightarrow y = c e^{2x},
\]

pero como digimos, la idea es ilustrar el m\'etodo.

El siguiente ejemplo no puede ser resuelto por las t\'ecnicas estudiadas hasta el momento, a pesar de 
ser muy simple en apariencia.

\vspace{0.5cm}

{\bf Ejemplo} \\
Usando series de potencias resuelva la ecuaci\'on diferencial $y^{\prime \prime} + x y^{\prime} + y = 0$

\vspace{0.5cm}

{\bf Soluci\'on} \\

\vspace{0.5cm}

Suponga que

\[
y =\sum_{n=0}^{\infty} a_n x^n
\]

es una soluci\'on de la ecuaci\'on diferencial. Entonces


\[
y^{\prime} = \sum_{n=0}^{\infty} n a_n x^{n-1} \Rightarrow xy^{\prime} = \sum_{n=0}^{\infty} n a_n x^n 
\]

y

\[
y^{\prime \prime} = \sum_{n=0}^{\infty} n \left(n - 1 \right) a_n x^{n-2}
\]

Sustituyendo en la ecuaci\'on diferencial

\begin{eqnarray*}
\sum_{n=0}^{\infty} n \left( n-1 \right) a_n x^{n-2} + \sum_{n=0}^{\infty} n a_n x^n + \sum_{n=0}^{\infty} a_n x^n & = & 0 \\
\sum_{n=0}^{\infty} n \left( n -1 \right) a_n x^{n-2} & = & -\sum_{n=0}^{\infty} \left(n + 1  \right) a_n x^n \\
\end{eqnarray*}

Ajustando los �ndices

\[
\sum_{n=-2}^{\infty} \left(n + 2 \right) \left(n + 1 \right) a_{n+2} x^n = - \sum_{n=0}^{\infty} \left(n + 1 \right) a_n x^n
\]

Igualando los coeficientes

\[
a_{n+2} = - \frac{n + 1}{\left(n+2 \right) \left(n + 1 \right)} a_n = - \frac{a_n}{n+2}
\]

para $n \geq 0$.

De esta forma los coeficientes de la serie soluci\'on est\'an
dados por:

\begin{itemize}

\item Coeficientes pares:

\begin{eqnarray*}
a_2 & = & - \frac{a_0}{2} \\
a_4 & = & - \frac{a_2}{4}= \frac{a_0}{2 \cdot 4} \\
a_6 & = & - \frac{a_4}{6} = - \frac{a_0}{2 \cdot 4 \cdot 6} \\
\vdots & & \vdots \\
a_{2n} & = & \frac{(-1)^n a_0}{2 \cdot 4 \cdot 6 \ldots (2n)} = \frac{(-1)^n a_0}{2^n n!}
\end{eqnarray*}

\item Coeficientes impares:

\begin{eqnarray*}
a_3 & = & - \frac{a_1}{3} \\
a_5 & = & - \frac{a_3}{5}= \frac{a_1}{3 \cdot 5} \\
a_7 & = & - \frac{a_5}{7} = - \frac{a_1}{3 \cdot 5 \cdot 7} \\
\vdots & & \vdots \\
a_{2n+1} & = & \frac{(-1)^n a_1}{3 \cdot 5 \cdot 7 \ldots (2n+1)}
\end{eqnarray*}

\end{itemize}


De esta forma la serie soluci\'on se puede representar como la
suma de dos series, una para las potencias pares con  coeficientes
en t\'erminos de $a_0$ y otra para las potencias impares con
coeficientes en t\'erminos de $a_1$.

\[
y = a_0 \sum_{n=0}^{\infty} \frac{(-1)^n x^{2n}}{2^n n!} + a_1 \sum_{n=0}^{\infty} \frac{(-)^n x^{2n+1}}{3 \cdot 5 \cdot 7 \ldots (2n+1)}
\]

{\bf Observaci\'on:} la soluci\'on tiene dos constantes
arbitrarias $a_0$ y $a_1$ tal como era de esperar para una
ecuaci\'on diferencial de segundo orden.

El siguiente ejemplo ilustra el procedimiento cuando la ecuaci\'on
diferencial tiene condiciones iniciales.

\vspace{0.5cm}

{\bf Ejemplo} \\


\vspace{0.5cm}

Use el teorema de Taylor\footnote{Si $f$ es derivable hasta el orden $n+1$ en un intervalo $I$ que contiene 
a $c$, entonces para toda $x$ en $I$ existe un $z$ entre $a$ y $c$ tal que

\[
f(x) = \sum_{n=0}^n \frac{f^{(n)}(c)}{n!} \left(x - c \right)^n + \frac{f^{(n+1)} \left(z \right)}{(n+1)!} \left( x - c  \right)^{n+1}
\]

}
para hallar la
soluci\'on en serie de potencias del problema de valor inicial

\[
\left\{
\begin{array}{rcl}
y^{\prime} & = & y^2 -x \\
y(0) & = & 1
\end{array}
\right.
\]

A continuaci\'on, use los primeros seis t\'erminos de la
soluci\'on para aproximar los valores de $y$ en el intervalo
$[0,1]$ con un paso de avance de $0.1$.

\vspace{0.5cm}

{\bf Soluci\'on} \\

\vspace{0.5cm}


La soluci\'on $y$ del problema de valor inicial puede expresarse por medio del teorema de Taylor como 

\[
y = y(0) + \frac{y^{\prime}(0)}{1!} x + \frac{y^{\prime \prime}(0)}{2!} x^2  + \frac{y^{\prime \prime}(0)}{3!} x^3 + \ldots
\]

con $c=0$. Como $y(0)=1$ y $y^{\prime} = y^2 - x$, tenemos que


\begin{eqnarray*}
y^{\prime} = y^2 -x & \Rightarrow & y^{\prime}(0) = 1 \\
y^{\prime \prime} = 2yy^{\prime} - 1 & \Rightarrow & y^{\prime \prime}(0) = 2-1=1  \\
y^{\prime \prime \prime} = 2yy^{\prime \prime} + 2 \left( y^{\prime} \right)^2 & \Rightarrow & y^{\prime \prime \prime}(0)=2+2=4 \\
y^{(4)} = 2yy^{\prime \prime \prime} + 6y^{\prime}y^{\prime \prime} &  \Rightarrow & y^{(4)} = 8 + 6 = 14 \\
y^{(5)} = 2yy^{(4)} + 8y^{\prime} y^{\prime \prime \prime} + 6 \left( y^{\prime \prime} \right)^2 & \Rightarrow & y^{(5)}(0) = 28 + 32 + 6 = 66 \\
\end{eqnarray*}

De donde obtenemos que

\[
y = 1 + x + \frac{1}{2} x^2 + \frac{4}{3!} x^3 + \frac{14}{4!} x^4 + \frac{66}{5!} x^5 + \ldots
\]

Usando los seis primeros t\'erminos de esta serie, calculamos los
valores de $y$ que se muestran en la siguiente tabla, adem\'as en
la figura \ref{serie1} se muestra la gr\'afica de este polinomio.

\begin{center}
\begin{tabular}{|c|c|c|c|} \hline
$x$ & $y$    & $x$ & $y$ \\ \hline
0   & 1      & 0.6 & 2.0424 \\ \hline
0.1 & 1.1057 & 0.7 & 2.4062\\ \hline
0.2 & 1.2264 & 0.8 & 2.8805 \\ \hline
0.3 & 1.3691 & 0.9 & 3.4985\\ \hline
0.4 & 1.5432 & 1   & 4.3 \\ \hline
0.5 & 1.7620 & 1.1 & 5.28999 \\ \hline
\end{tabular}
\end{center}

{\bf Observaci\'on:} para una serie de Taylor entre m\'as lejos estemos del centro de convergencia 
(en este caso $c=0$), menor es la precisi\'on de nuestra estimaci\'on. Es importante tener claro que si 
las condiciones iniciales est\'an dadas en $x=a$, debemos usar el desarrollo en series de potencias para la
soluci\'on $y$ alrededor de $c=a$.


\begin{figure}
\begin{center}
\includegraphics{serie1.eps}
\end{center}
\label{serie1}
\caption{Gr\'afica de $y(x)$}
\end{figure}



\vspace{0.5cm}

{\bf Ejemplo}\\

\vspace{0.5cm}

Encuentre una series de potencias para la soluci\'on general de la
ecuaci\'on diferencial

\[
y^{\prime \prime} + Sen(x) y^{\prime} + e^x y = 0
\]


{\bf Soluci\'on}\\

\vspace{0.5cm}
Como todos los coeficientes admiten desarrollo alrededor de $x=0$,
podemos suponer que la soluci\'on $y$ es de la forma

\[
y = a_0 + a_1x + a_2x^2 + a_3x^3 + \ldots + a_n x^n + \ldots = \sum_{n=0}^{\infty} a_n x^n
\]

Adem\'as, recuerde que

\[
Sen(x) = x - \frac{1}{3!} x^3 + \frac{1}{5!} x^5 + \ldots + \frac{(-1)^n}{(2n+1)!} x^{2n+1}  = \sum_{n=0}^{\infty} \frac{(-1)^n}{(2n+1)!} x^{2n+1}
\]

y

\[
e^x = 1 +x + \frac{1}{2!} x^2 + \frac{1}{3!} x^3 + \ldots + \frac{1}{n!} x^n + \ldots = \sum_{n=0}^{\infty} \frac{1}{n!}x^n
\]

para toda $x \in \R$.

Sustituyendo en la ecuaci\'on diferencial obtenemos que

\begin{eqnarray*}
\sum_{n=0}^{\infty} n\left( n-1 \right)a_n x^{n-2} + \left( \sum_{n=0}^{\infty} \frac{(-1)^n}{(2n+1)!} x^{2n+1}  \right) \left( \sum_{n=0}^{\infty} n a_n x^{n-1}\right) & + & \\
 \left(  \sum_{n=0}^{\infty} \frac{1}{n!}x^n \right) \left( \sum_{n=0}^{\infty} a_n x^n  \right) & = & 0
\end{eqnarray*}


Multiplicando las series y simplificando tenemos que

\begin{eqnarray*}
\left(2a_2 + a_0  \right) + \left(6a_3 + 2a_1 + a_0 \right)x + \left(12a_4 + 3a_2 + a_1 + \frac{a_0}{2}  \right)x^2 & + & \\
\left(20a_5 + 4a_3 + a_2 + \frac{a_1}{2} + \frac{a_0}{6} \right) x^3 + \ldots & = & 0
\end{eqnarray*}

Igualando cada uno de los coeficientes a cero

\begin{eqnarray*}
a_2 & = & - \frac{a_0}{2} \\
a_3 & = & - \frac{2a_1 + a_0}{6}=- \frac{a_1}{3} - \frac{a_0}{6} \\
a_4 & = & - \frac{a_2}{4} - \frac{a_1}{12} - \frac{a_0}{24} = \frac{a_0}{8} - \frac{a_1}{12} - \frac{a_0}{24} = \frac{a_0}{12} - \frac{a_1}{12} \\
a_5 & = & - \frac{a_3}{5} - \frac{a_2}{20} - \frac{a_1}{60} - \frac{a_0}{120} = \frac{a_0}{20} + \frac{a_1}{20}
\end{eqnarray*}


Sustituyendo estos valores en la serie tenemos que

\[
y = a_0 \left(1 - \frac{1}{2} x^2 - \frac{1}{6} x^3 + \frac{1}{12} x^4 + \frac{1}{20} x^5 + \ldots   \right) + a_1 \left(   x - \frac{1}{3} x^3 - \frac{1}{12} x^4 + \frac{1}{20} x^5 + \ldots \right)
\]

para $x \in \R$.


{\bf Observaci\'on:} algunas veces cuando necesitamos multiplicar
dos series es \'util la siguiente f\'ormula

\[
\left( \sum_{n=0}^{\infty} a_n x^n \right) \left( \sum_{n=0}^{\infty} b_n x^n  \right) = \sum_{n=0}^{\infty} \left( \sum_{k=0}^{n} a_k b_{n-k}\right) x^n
\]


\end{document}
